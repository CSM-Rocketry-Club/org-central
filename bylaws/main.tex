\documentclass[12pt]{article}
\usepackage{times}
\usepackage{sectsty}
\renewcommand{\thesection}{Article \Roman{section}}
\renewcommand{\theparagraph}{(\arabic{paragraph})}
\renewcommand{\thesubparagraph}{(\alph{subparagraph})}
\setcounter{secnumdepth}{9}
\usepackage{todonotes}
\usepackage[letterpaper, margin = 1.5in]{geometry}
\usepackage{color, soul}
\usepackage{enumitem}
\usepackage{cleveref}
\begin{document}
\section{Name of the Organization} 
The name of the organization here described shall officially be called The Rocket Club, which may also be referred to as Rocket Club.

\section{Purpose}
The purpose of this club shall be to design and build model rockets from raw materials or from kits in a safe and professional manner. 
The goal is to excite Mines students about aerospace engineering and the opportunities for aerospace work in their own majors. 
Club members will work on their own individual rockets in pursuit of certification from the Tripoli Rocketry Association. 
Members will launch their completed rockets at Tripoli Rocketry Association approved launch sites.

\section{Membership}
Membership is strictly limited to students of Mines according to the budget policies laid forth at the inception of the student activities fee.

\paragraph{Eligibility}
Membership is strictly limited to students of Mines according to budget policies laid forth at the inception of the student activities fee. 
Student money is mandated to be spent on students and students alone, 
with the exception of the faculty advisor and instructors, 
teachers and speakers the club utilizes. 
However, membership can be granted to non-students under specific rules. 

\paragraph{Privileges of Membership}
Members of the rocket club will be able to work on rocket and rocket part designs and participate in hands-on construction.  
They will also be able to attend and participate in launches.  
Attendance at meetings will be restricted to members and their invited guests.

\paragraph{Revocation of Membership}
Members found to be following unsafe practices during rocket construction or launch will be warned and repeat offenders will be removed from the club.  
Those being disruptive during meetings or found intentionally acting contrary to the spirit and purpose of the club will be treated in the same manner. 
Membership must be revoked by SAIL/Mines in accordance with the Colorado School of Mines Code of Conduct, Student Organizations Handbook, or any other pertinent school policies.

Regular members are those who have shown a vested interest in the club as per the officers' discretion, 
and consistently attend club meetings. 
\section{Officers}
\label{officers}
The officers of the organization shall consist of president, 
vice president, 
treasurer,
secretary, 
safety officer, 
outreach officer, 
historian, 
quartermaster, 
and guru. 
All officers will be provided Blastercard access to the lab space and shall hold officer privileges on the club Engage page.

\paragraph{Qualifications}
All officers shall be members of the organization who have been in good standing with the club for the duration of their membership.  
The position of president will require the candidate to have previously held a position in the organization, 
unless no such candidate exists. 
Members may only hold more than one officer position if there are not enough candidates to fill all positions. 
In the event that there are multiple candidates for the same officer position in an election, 
and one or more of the candidate(s) are Tripoli certified, 
then only those candidate(s) who are Tripoli certified shall qualify for the electoral ballot and receive any votes for that officer position.

\paragraph{Duties}
\subparagraph{President} The president shall perform duties customarily pertaining to the office; 
shall preside over meetings of the organization, 
send out major organization emails, 
and shall perform other duties as the organization shall assign.  
The president will also maintain all SAIL and BSO relations between the club and the Colorado School of Mines in order to ensure the club's continued funding. 
The president shall hold a first club meeting at the beginning of the academic year to introduce new members to the club. 
The president shall make the utmost effort to assist other members with their Rocket Club projects, 
ensure the continued success of the club, and meet all funding and allocation requirements.

\subparagraph{Vice President} The vice president shall perform duties customarily pertaining to the office; 
shall preside over meetings of the organization in the president's absence and shall perform duties assigned by the president. 
Vice president shall also act as a back-up treasurer when this service is needed. 
The vice president will be responsible for maintaining contact with the club advisor and will work closely with the club quartermaster to keep the workspace maintained. 

\subparagraph{Quartermaster} The quartermaster will be responsible for maintenance of the club's facilities, tools, and materials. 
The quartermaster will also work with the lab managers to ensure the Rocket Club retains a fair share of the lab space.

\subparagraph{Treasurer} The treasurer shall perform duties customarily pertaining to the office; 
shall keep an accurate ledger of the organization's budget and expenditures, 
shall report to the treasurer of BSO, 
shall prepare a budget preceding the end of the spring semester deadline for the submittal to the BSO financial committee, 
shall disburse the budget of the current fiscal year under BSO financial guidelines, 
shall perform any duties assigned by the president, and shall meet any other BSO requirements not here stated. 
The treasurer must attend treasurer training when required by BSO. 
The treasurer is also responsible for regularly procuring, in a timely manner, 
materials and other supplies for use by the club. 

\subparagraph{Secretary} The secretary shall perform duties customarily pertaining to the office; 
shall act as a secretary of all the administrative meetings of the organization and record the minutes thereof, 
shall, under supervision of the president, conduct all official correspondence, 
and keep a roll of all members of the organization (in the event this proves necessary), 
and shall perform those duties assigned by the president. 
The secretary will also maintain flight records and experimental data for the club, 
as well as the club's Engage page and OneDrive.

\subparagraph{Safety Officer} The safety officer shall perform duties pertaining to the safety of the club as a whole and the well-being of the workspace, 
and ensure that all members are familiar with safe operating practices of all major equipment in the lab. 
The safety officer shall obtain any and all required safety training by the Colorado School of Mines, SAIL, and/or BSO. 
The safety officer shall also have the power, in addition to the president, 
to prohibit members from using certain tools or, if necessary, 
ban them lab access if they consistently follow unsafe lab practices.
\hl{
The safety officer shall also have the power to review, and require the modification of, any airframe designs that are deemed unsafe or unconventional.
}
\todo{Actions that may be taken by the saftey officer pursuant to this rule will be in policy.}

\subparagraph{Outreach Officer} The outreach officer shall perform duties pertaining to the organization and running of club events, 
including Celebration of Mines and all other service events as required by BSO. 
The outreach officer shall also perform duties pertaining to the recruitment and maintenance of club members. 
The outreach officer will be responsible for organizing and completing a minimum of three community service events per school year as required to maintain Tier 3 status with the BSO. 
The outreach officer will also collaborate with the president and quartermaster to conduct tours of the lab to new members and visitors.

\subparagraph{Club Historian} The historian will be responsible for photographing important club activities, 
including launches, major rocket construction, and motor tests. 
The historian shall further be responsible for uploading media to the club OneDrive and Engage page.

\subparagraph{Guru} The guru will be responsible for instructing new members and answering questions about the science and practice of rocketry. 
They will also be responsible for maintaining all club instructional and informational documents such as slide decks, 
video and written tutorials, Solidworks template files, etc.

\paragraph{Vacancies}
In the event of a vacancy in a club position, 
whether by resignation from the position, 
leaving Mines, suspension or expulsion from Mines, 
any other situation requiring their removal by Mines, 
by impeachment and removal from office via agreement of all other non-vacant officers, 
or any other reasonable situation, 
then the officer position will be filled via appointment by the president for the duration of the academic year until a new officer is elected to that position and their term begins. 
In the event that there is a vacancy in the president position, 
the vice president shall become the president who will then appoint a new vice president. 
SAIL Staff must be involved in the removal process of anyone from an officer or membership removal.

\paragraph{Advisor}
A faculty advisor is required by the Colorado School of Mines. 
An Advisor may apply or be appointed by the organization. 
Advisor may be re-evaluated by the club yearly. 
Advisors have no voting powers. 
Required-Advisors must fill out the Advisor Agreement and be CSA Trained. 
Advisors shall act as primary contact in addition to the Faculty Advisor for the campus community and administration regarding the organization. 

\paragraph{Alumni Advisor}
Alumni Advisors are considered Mines Alumni who are no longer undergraduate students at the Colorado School of Mines and whose primary role is to support the wellbeing of the organization. 
Graduate students can serve as an Alumni Advisor if they no longer wish to be a student participant in the organization. 
An Alumni Advisor may apply or be appointed by the organization. Alumni Advisors have a one-year term limit with the option for renewal and must live in Colorado. 
Alumni Advisors have no voting powers. Alumni Advisors must fill out the Advisor Agreement and be CSA Trained. 
Alumni Advisors shall act as primary contact in addition to the Faculty Advisor for the campus community and administration regarding the organization. 
Alumni Advisors shall be responsible for publicizing the organization in a productive manner to the school and community. 
Alumni Advisors shall act as primary contact, aid, and advocate for their organization. 
Alumni Advisors shall have general oversight of the organizations, and be familiar with university, governing body, and SAIL processes to best aid their organization.

\section{Meetings}
\paragraph{Calling of Meetings}
The club will hold two types of meetings: 
general lab meetings during which club members will have the opportunity to work on rockets and prepare for launches, 
and administrative meetings, 
at which the officers shall meet to plan the future goals and efforts of the organization.
The president shall call an administrative meeting whenever required for the conduct of official business. 
General meetings shall be held weekly for working on/starting rocket projects, 
or setting dates and times for launches or preparing for launches. 
It is recommended that regular meeting times be established and communicated with all club members. 
General meetings will also be referred to as labs. 
Labs are non-required times for club members to come to the club's meeting location to work on their Rocket Club projects. 
\paragraph{Quorum}
Two-thirds of the \hl{Officers} of the organization shall constitute a quorum for the transaction of official business. 
A quorum will be assumed to be present unless a roll call vote is called for by an officer.

\section{Election Process}
Officer Elections will be conducted once a year in March, or earlier if required, 
prior to the submittal of the yearly required form from SAIL. 
Newly elected officers will not formally hold the position until the end of the current academic year. 
Between that time new officers will be trained by the current officers for the position.
\todo{No provision to prevent running for all positions. Should this get fixed?}{Election Procedures:}
\begin{enumerate}[label=(\roman*)]
    \item An in-person election will be held over two subsequent regular club meetings in March, 
    or earlier if required, prior to club form submittal. 
    \item An email will be sent one month before the first day of the elections, most likely at the beginning of February, 
    to the club members to tell them the election will take place and encourage candidates to run. 
    A reminder will be sent out one week in advance and another one day in advance of the first day of the election.
    \item Candidates must nominate themselves. 
    All qualifications must be met as described in Article IV, Section B of these bylaws. 
    \item If a potential qualified candidate cannot attend the election, 
    then they may email the current president and the current vice president their intent to run, 
    position they intend to run for, and the reason why they cannot attend. 
    The president will ensure that their name is added to the ballot and the vice president will ensure that it is done accurately.
    \item The election will be held in person where attendants will cast their votes on one club-combined ballot. 
    Voters will put a tally next to the name of the candidate for each position they wish to vote for. 
    Each voter may only vote once per position but may vote for all positions. 
    This will be done in a public setting. 
    Voters may only change or remove their vote if it is certain by all other voters that they did in fact vote for the tally that they are removing their vote from.
    \item If a regular member cannot attend the election, 
    they may email the current president and vice president their desire to vote. 
    The president will respond notifying them of the candidates for each position in a timely manner. 
    The regular member then may email the current president and vice president the candidates they wish to vote for. 
    The president will ensure that their votes are counted on the ballot and the vice president will ensure their accuracy.
    \item The completed ballot will be publicly displayed after the election is completed. 
    Whichever candidate receives a plurality of the votes cast will become the officer-elect for that position. 
    If there are more than two candidates for the same position and no one candidate received over 50\% of the total votes cast, 
    then the president may call a runoff election instead if deemed necessary. 
    If there is a tie, then the current officers will decide among themselves who wins the election.
    \item The new officers will be communicated to SAIL as required.
\end{enumerate}
To continue with club operations, it is critical that a President, Treasurer, and Safety Officer be elected. 
Other officer positions shall be filled only if there is a qualified candidate for each of these three positions.

\section{Operational Administration}
\todo{Pending Approval}{
    \paragraph{Policies}
    The leadership of The Rocket Club shall maintain a set of policies to ensure:
    \begin{enumerate}[label=(\roman*)]
        \item safe operation during lab meetings,
        \item safe operation during launches,
        \item effective utilization of club funding,
        \item cleanliness of the workspace, and,
        \item a positive and encouraging atmosphere.
    \end{enumerate}
    \todo{Should this be a policy instead?}{
    \subparagraph{{\it \bf Ad-Hoc} Safety Policies}
    In the event that the safety officer directly observes unsafe behaviors, an {\it Ad-Hoc} policy may be created.
    These policies are enforceable from time of inaction to the next meeting of the club, 
    where it shall undergo the same policy review process as any other policy.
    }
}
\section{Non-Discrimination}
\todo{Update to current language}{
The organization, in conjunction with the Colorado School of Mines, 
confirms its support of the principles and practices of nondiscrimination and equality regardless of race, 
religion, sex, age, sexual orientation or physical handicap, in its employment and in all of its programs, activities, 
and opportunities available to its members, except where allowed by law. All actions and policies of the organization, 
Associated Students of the Colorado School of Mines, 
and the Colorado School of Mines Graduate Student Association shall conform to the policies of the Board of Trustees, 
State and Federal Laws, such as 20 USC §1681 and agree to support SAIL's policies, processes, and guidance.
}

\section{Amendments}
These bylaws are subject to initial approval by Mines, after submittal to and subject to revision by the Student Affairs Committee. 
Following the initial approval by Mines, these bylaws may be revised at any time by a vote of at least two-thirds (2/3) of the active members of the organization in good standing (article III C.), 
with regard to state law, the student code of conduct, the bylaws of the Colorado School of Mines, and the Mines Budget Committee operating rules. 
The bylaws may also be amended by a consensus of all the club officers \todo{Review}{as defined by \ref{officers}}.
\end{document}